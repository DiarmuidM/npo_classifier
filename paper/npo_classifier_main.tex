\documentclass[11pt]{article}
\usepackage[letterpaper, left=1in,top=1in,right=1in,bottom=1in]{geometry}
\usepackage[usenames, dvipsnames]{color}
\usepackage[style=apa]{biblatex}
\usepackage{graphicx}
\usepackage{hyperref}
\usepackage{longtable}
\usepackage{multirow}
\usepackage{bibentry}
\usepackage{todonotes}
\usepackage{graphicx}
\usepackage{arydshln}
\usepackage[symbol]{footmisc}

\renewcommand{\baselinestretch}{1.5}
\renewcommand{\thefootnote}{\fnsymbol{footnote}}
\bibliography{C:/Users/Ji/Zotero/zotero.bib}
\setcounter{tocdepth}{2}

\begin{document}

\begin{center}
{\Large \textsc{An international nonprofit classification system: A machine-learning approach experimented on China, United Kingdom, and United States}}
\end{center}

The National Taxonomy of Exempt Entities (NTEE) has been used for classifying the nonprofit organizations in the United States for several decades. However, major countries in the world do not have a classification system for the nonprofit sector. This paper achieves three major goals: 1) devising a machine learning model which can classify the nonprofits using mission statements, 2) inventing a functional classification system which can be applied to different countries, 3) test the accuracy of the model and classification system. We first created a classification system cross different countries by matching existing standards, then compiled the training and testing datasets for China (data from China Foundation Center and Research Infrastructure of Chinese Foundations), United Kingdom (data from ****), and United States (data from National Center for Charitable Statistics and Internal Revenue Service). We finally test the accuracy of major text classification algorithms using country-specific training datasets and a pooled dataset. Implications and limitations are discussed.

\end{document}